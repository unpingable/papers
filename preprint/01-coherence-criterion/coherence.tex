\documentclass[11pt]{article}
\usepackage{amsmath}
\usepackage{amssymb}
\usepackage{authblk}

% Preamble is clean. Subtitle is folded into the title.
\title{\textbf{The Coherence Criterion: A Unified Framework for Stability in Hierarchical Systems}\\[0.5em]
{\normalsize Temporal Coupling, Spectral Constraints, and Cross-Domain Failure Modes}}
\author[1]{James Beck}
\affil[1]{Independent Researcher}
\date{November 2025}

\begin{document}

\maketitle

\begin{abstract}
The \textbf{Coherence Criterion} proposes a unified mathematical framework for system stability in hierarchical systems operating across multiple temporal scales ($\Delta t$). System viability depends on maintaining bounded temporal divergence between coupled layers, formalized via \textbf{spectral analysis of coupled linear operators} \cite{strogatz2015}.

The stability condition is $\rho(M) < 1$ (where $\rho$ is the spectral radius of the coupling matrix $M$). Violation of this criterion leads to two pathological regimes: \textbf{Rigidity Runaway} ($\lambda \to +1$) and \textbf{Acceleration Runaway} ($\lambda \to -1$) \cite{strogatz2015}.

The paper argues this represents \textbf{structural correspondence} across autonomous vehicles, LLMs, institutions, governance, and financial markets, with failures exhibiting qualitatively similar bifurcation geometry \cite{strogatz2015}.
\end{abstract}

% -------------------------------------------------------------------
\section{The Coherence Criterion - Theory}
% -------------------------------------------------------------------

\subsection{The Central Claim}
The fundamental constraint for system survival is that internal tempos must stay coupled tightly enough to maintain coherence \cite{murray2014}. The central claim is that \textbf{temporal structure is the organizing principle of complex systems} \cite{murray2014}.

\begin{itemize}
    \item \textbf{Failure Mechanism:} Failure occurs when a \textbf{fast layer decouples from a slower integration layer} because the temporal gap exceeds the system's coupling bandwidth \cite{kirilenko2017}.
\end{itemize}

\begin{table}[h!]
\centering
\small % Use small font for the table
\caption{Cross-Domain Examples of Temporal Layering Failure}
% Reduced p{width} for the last column to p{3cm}
\begin{tabular}{|l|c|c|p{3cm}|}
\hline
\textbf{Example} & \textbf{Fast Layer ($\tau_1$)} & \textbf{Slow Layer ($\tau_2$)} & \textbf{Resulting Failure} \\
\hline
Autonomous Vehicle & Perception ($\sim$100ms) & Tactical Planning ($\sim$1s) & No consistent world model, Fatal Collision (2018) \\
\hline
2010 Flash Crash & HFT Algorithms (ms) & Human Traders (s-min) & Market lost \$1 trillion \\
\hline
LLM Hallucination & Token Generation (ms) & Semantic Verification (External) & Locally plausible, globally false text \cite{goodfellow2016} \\
\hline
\end{tabular}
\label{tab:examples}
\end{table}

\subsection{Mathematical Framework}
The evolution of two coupled layers ($z_1$, $z_2$) is modeled by a discrete-time system $z_{t+1} = M z_t + \eta_t$, where $M$ is the coupling matrix:
\begin{equation}
M = \begin{pmatrix} a_1 & c_{12} \\ b_{21} & a_2 \end{pmatrix}
\end{equation}

\begin{itemize}
    \item $a_k \in [0,1)$: Layer persistence/damping \cite{strogatz2015}.
    \item $c_{12}$: Fast layer response to slow layer (strategic guidance) \cite{strogatz2015}.
    \item $b_{21}$: Slow layer integration of fast layer (evidence accumulation) \cite{strogatz2015}.
\end{itemize}

\textbf{Theorem 1 (Coherence Criterion):} The system maintains bounded trajectories if and only if the spectral radius $\rho(M) < 1$.

The Eigenvalues ($\lambda$) are given by:
\begin{equation}
\lambda = \frac{a_1 + a_2}{2} \pm \sqrt{\frac{(a_1 - a_2)^2}{4} + b_{21}c_{12}}
\end{equation}

\textbf{Critical Boundaries:}
\begin{enumerate}
    \item \textbf{Divergence Boundary ($\lambda \to +1$):} Coupling exceeds damping, $b_{21}c_{12} > (1 - a_1)(1 - a_2)$ \cite{strogatz2015}.
    \begin{itemize}
        \item \textbf{Failure Mode:} \textbf{Rigidity Runaway} (positive feedback, locked-in states) \cite{strogatz2015}.
    \end{itemize}
    \item \textbf{Flip Boundary ($\lambda \to -1$):} Negative feedback dominates, $b_{21}c_{12} < -(1 + a_1)(1 + a_2)$ \cite{strogatz2015}.
    \begin{itemize}
        \item \textbf{Failure Mode:} \textbf{Acceleration Runaway} (oscillatory instability, over-correction) \cite{strogatz2015}.
    \end{itemize}
\end{enumerate}

\subsection{The Coherence Metric ($\chi$)}
This metric measures coupling bandwidth relative to temporal divergence:
\begin{equation}
\chi = \frac{b_{21}c_{12}}{|a_1 - a_2|}
\end{equation}
\begin{itemize}
    \item Coherence requires $0 < \chi < \chi_{critical}$ AND $\rho(M) < 1$ \cite{strogatz2015}.
\end{itemize}

\subsection{The Principle of Temporal Adjacency}
Hierarchy emerges as the evolutionary solution to the $\Delta t$ problem \cite{murray2014}.

\begin{itemize}
    \item \textbf{Principle:} A coordination system can maintain coherence only when interacting layers differ by no more than \textbf{O($10^2$) in characteristic timescale ($\kappa < 100$)} \cite{murray2014}.
    \item When $\kappa > 100$, the coupling strength required to maintain stability pushes eigenvalues toward the critical boundaries \cite{murray2014}.
    \item This limit appears consistently across control systems, neural systems, economic systems, and social coordination \cite{murray2014}.
\end{itemize}

% -------------------------------------------------------------------
\section{The Coherence Map - Cross-Domain Validation}
% -------------------------------------------------------------------
The framework is validated by \textbf{Stress Equivalence Mapping}, which tests if different systems exhibit \textbf{Qualitatively Equivalent Eigenvalue Trajectories} and \textbf{Phenomenological Correspondence} at critical boundaries under stress \cite{strogatz2015}.

\begin{table}[h!]
\centering
\footnotesize % Use smallest standard font size (guaranteed fit)
\caption{Cross-Domain Failure Mode Correspondence}
% Aggressively reduced p{width} for the three rightmost columns
\begin{tabular}{|c|c|p{1.8cm}|p{2.2cm}|p{2.5cm}|}
\hline
\textbf{Boundary} & \textbf{Failure Mode} & \textbf{AV Example} & \textbf{Institutional Example} & \textbf{Financial Example} \\
\hline
$\lambda \to +1$ & \textbf{Rigidity Runaway} (Slow Dominance) & Wrong object classification persists (Uber) & Policy paralysis, Institutional lockup (Governance) \cite{lewis2014} & Market rigidity before crashes (2008 Crisis) \cite{lewis2014} \\
\hline
$\lambda \to -1$ & \textbf{Acceleration Runaway} (Fast Decoupling) & Phantom braking on transient signals \cite{ntsb2019} & Policy whiplash, Overreaction (Governance) \cite{lewis2014} & Flash crash, Algorithmic oscillation (2010 Crash) \cite{cftcsec2010} \\
\hline
\end{tabular}
\label{tab:failure_modes}
\end{table}

\subsection{Case Study Highlights}
\begin{itemize}
    \item \textbf{Academic Publishing:} The Replication Crisis is a $\Delta t$ failure. Research ($\tau_1$, weeks) is decoupled from legitimacy verification ($\tau_3$, years), a gap $\approx 10^2$ \cite{osc2015}. Increasing publication velocity ($b_{21} \uparrow$) without accelerating verification ($\tau_3$ constant) pushes the system toward instability \cite{osc2015}.
    \begin{itemize}
        \item \textbf{Interventions validated:} Pre-registration (reduces fast-layer gain $b_{21}$) and replication studies (strengthens $\tau_3$ verification) \cite{osc2015}.
    \end{itemize}
    \item \textbf{Governance Collapse:} The temporal divergence between social media ($\tau_1$, hours) and constitutional norms ($\tau_3$, decades) increased from $\approx 10^3$ to $\approx 10^5$ between 1970 and 2020, while coupling remained constant \cite{lewis2014}. This puts the system near the coherence envelope, causing policy oscillation and rigidity (e.g., January 6th, policy whiplash) \cite{lewis2014}.
    \item \textbf{Financial Markets:} The shift to High-Frequency Trading (HFT) increased temporal divergence between trading ($\tau_1$) and human oversight ($\tau_2$) by \textbf{six orders of magnitude} \cite{kirilenko2017}.
    \begin{itemize}
        \item \textbf{2010 Flash Crash} was \textbf{Acceleration Runaway} ($\lambda \to -1$): fast layer decoupled, oscillating without damping \cite{cftcsec2010}.
        \item \textbf{2008 Financial Crisis} was \textbf{Rigidity Runaway} ($\lambda \to +1$): slow regulatory frameworks persisted in outdated models despite fast-layer evidence of danger \cite{lewis2014}.
    \end{itemize}
\end{itemize}

% -------------------------------------------------------------------
\section{Implications and Predictions}
% -------------------------------------------------------------------

\subsection{Practical Implications for Design}
The Coherence Criterion provides clear design principles for multi-layer systems:
\begin{enumerate}
    \item \textbf{Respect Adjacency:} Do not couple layers separated by more than O($10^2$) in timescale without explicit integration \cite{murray2014}.
    \item \textbf{Monitor Stress:} Track gain amplification, latency spikes, and oscillation as early warning signs \cite{strogatz2015}.
    \item \textbf{Intervention Hierarchy:} When coherence fails, the fix is always structural \cite{kirilenko2017}:
    \begin{itemize}
        \item Reduce $\Delta t$ between layers.
        \item Strengthen coupling mechanisms.
        \item Add mid-layer integration where missing (e.g., Retrieval-Augmented Generation in LLMs, which the framework correctly predicts will reduce hallucination) \cite{goodfellow2016}.
    \end{itemize}
\end{enumerate}

\subsection{Falsifiable Prediction}
The framework generates testable predictions \cite{strogatz2015}:
\begin{itemize}
    \item \textbf{General:} Measures that reduce $\Delta t$ or adjust coupling should restore stability \textbf{regardless of domain} \cite{strogatz2015}.
    \item \textbf{Academia:} Fields with higher publication velocity (larger $b_{21}$) and weaker verification ($\tau_3$ years) should exhibit higher replication failure rates \cite{osc2015}.
    \item \textbf{Governance:} Continued temporal divergence will require either adaptation of slow-layer mechanisms (compress $\tau_3$) or fragmentation (accept decoupling) \cite{lewis2014}.
\end{itemize}

% -------------------------------------------------------------------
\section*{References}
% -------------------------------------------------------------------
\begin{thebibliography}{9}
    \bibitem{ntsb2019}
    National Transportation Safety Board. (2019). \textit{Collision Between Vehicle Controlled by Developmental Automated Driving System and Pedestrian, Tempe, Arizona, March 18, 2018.} Highway Accident Report NTSB/HAR-19/03. Washington, DC.
    
    \bibitem{cftcsec2010}
    U.S. Commodity Futures Trading Commission \& U.S. Securities and Exchange Commission. (2010). \textit{Findings Regarding the Market Events of May 6, 2010.} Joint Report. Washington, DC.
    
    \bibitem{murray2014}
    Murray, J. D., Bernacchia, A., Freedman, D. J., Romo, R., Wallis, J. D., Cai, X., ... \& Wang, X. J. (2014). A hierarchy of intrinsic timescales across primate cortex. \textit{Nature Neuroscience}, 17(12), 1661-1663.
    
    \bibitem{osc2015}
    Open Science Collaboration. (2015). Estimating the reproducibility of psychological science. \textit{Science}, 349(6251), aac4716.
    
    \bibitem{strogatz2015}
    Strogatz, S. H. (2015). \textit{Nonlinear Dynamics and Chaos: With Applications to Physics, Biology, Chemistry, and Engineering} (2nd ed.). Westview Press.
    
    \bibitem{kirilenko2017}
    Kirilenko, A., Kyle, A. S., Samadi, M., \& Tuzun, T. (2017). The flash crash: High-frequency trading in an electronic market. \textit{The Journal of Finance}, 72(3), 967-998.
    
    \bibitem{lewis2014}
    Lewis, M. (2014). \textit{Flash Boys: A Wall Street Revolt}. W. W. Norton \& Company.
    
    \bibitem{goodfellow2016}
    Goodfellow, I., Bengio, Y., \& Courville, A. (2016). \textit{Deep Learning}. MIT Press.
\end{thebibliography}

\end{document}
